\documentclass{article}
\usepackage[utf8]{inputenc}
\usepackage[spanish]{babel}
\title{Modelos matemáticos discretos}
\author{Verania Lizbeth Pérez Robles}
\begin{document}
\maketitle
\section{Ecuaciones en diferencias}
\subsection{Primer orden}

La ecuación $x_{n+1}=ax_n+b$ se llama una ecuación en diferencias de primer orden lineal con coeficientes no homogeneos.

Sabemos que $$\lim_{x\to\infty}\frac{1}{x}=0$$.

Tenemos \$1000 que vamos a invertir a un interés del 1\% mensual.
El valor de la inverión cuando han transcurrido $n$ meses es $$x_n=1000(1.01)^n$$.

Para enontrar este resultado, ocupamos que:
$$\sum_{i=0}^{n-1}a^{i}=\frac{1-a^{n}}{1-a}$$.
\subsection{Segundo orden}


\end{document}

